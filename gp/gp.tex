\documentclass{article}
\usepackage{amsmath}

\begin{document}
\title{Gaussian Processes}
\author{Kinh Tieu}
\date{August 28, 2017}
\maketitle

\section{Introduction}

A Gaussian process (GP) extends Gaussian distributions to functions.  Data from any finite subset of the range of a GP follows a Gaussian distribution.

\begin{equation}
y = f(x) + \mathcal{N}(0,\sigma^2_n)
\end{equation}

A GP is assumed to be zero mean with a covariance function:
\begin{equation}
k(x,x_i) = \sigma^2_f \exp \left( \frac{-(x-x_i)^2}{2h^2} \right) + \sigma^2_n \delta(x,x_i)
\end{equation}

\begin{equation}
\begin{pmatrix}
\mathbf{y}_{\cdot} \\
y
\end{pmatrix}
\tilde \mathcal{N}\left(0,
\begin{pmatrix}
K_{\cdot\cdot} & K^T_{\cdot} \\
K_{\cdot} & K
\end{pmatrix}
\right)
\end{equation}
\end{document}
